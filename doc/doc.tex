\documentclass{llncs}
%
\usepackage{makeidx}  % allows for indexgeneration
\usepackage[utf8]{inputenc}
\begin{document}
\mainmatter              % start of the contributions
%
\title{Study of Connection Patterns in Opportunistic Networks}
%
\author{Antonio Cardace}
%
\institute{Universit\`a di Bologna, Dipartimento di Scienze dell'Informazione
\email{antonio.cardace2@studio.unibo.it}\\Matricola : 0000738443}
\maketitle
%
\begin{abstract}
A mobile ad hoc network (MANET) is a continuously self-configuring, infrastructure-less network of mobile devices connected without wires. Each   device/node in a MANET is free to move independently in any direction, and   will therefore change its links to other devices frequently. The main problem in a MANET is to maintain connectivity. A simulation model has been implemented in NetLogo to analyze the behavior of the network according to certain varying properties, the results comprehend some common evaluation metrics calculated on many experiments in which different strategies for handling connections have been used.
\end{abstract}

\section{The Model}
\subsection{Base Model}
The model which has been used defines a type of agent : a node, every node moves independently and randomly. Two nodes can communicate when there is a link between them, this link gets created only if the distance between the two nodes is less or equal than $r$ which is the coverage area of a node. Obviously when two connected nodes move beyond their coverage area, the link between them is removed.\\\\
%
In addition to the coverage area of a node there is another constraint to be satisfied in order for a link to be established, both nodes must have a degree which is less than the maximum possible degree.\\\\
%
A node also has a defined speed of movement, according to this parameter we know how many steps the node can move forward in just a unit of time.\\\\
%
The last property of the model is the strategy the nodes adopt for handling connection, which defines the behavior when events such as "establish new connection" or "replace a connection" occur.\\\\
%
The area on which this simulation model has been deployed is a torus of variable size.\\
%
\subsection{Extensions to the Base Model}
The base model which has just been described has been extended with few properties.\\
\begin{itemize}
	\item \textbf{Node speed} : every node can have its own speed parameter. When the global parameter $s$ (node speed) is set a node set its own speed parameter which is chosen randomly and uniformly between 1 and $s$;\\
	%
	\item \textbf{Node maximum degree} : every node can have its own maximum degree parameter. When the global parameter $D_{m}$ (maximum degree) is set a node set its own maximum degree parameter which is chosen randomly and uniformly between 1 and $D_{m}$.
\end{itemize}
%
\subsection{Connection Handling Strategies}
%
The base rule is that every node establish a new connection when possible, therefore the different strategies define how a node behaves when it has reached the maximum number of connections and a new node gets discovered (it means that the new node enters in its coverage area), this implies that some decision has to be made about whether removing an older connection in order to establish the new one and according to what criterion this must be done.\\\\
%
These are the different connection handling strategies present in the NetLogo implementation of the model and which are therefore available for the simulation experiments:
\begin{enumerate}
	\item \textbf{Random-kill} : 
\end{enumerate}
\section{Evaluation Metrics}
For the purpose of evaluating the simulation results of the implementation of the model some metrics have been considered and computed on each performed experiment.\\
%
\subsection{Growth of Connected Component}
\subsection{Connectivity}
\subsection{Edge Density}
\subsection{Bridges}
\subsection{Degree Distribution}
%
\section{Experiments}
%
\section{Conclusions}
% ---- Bibliography ----
%
\begin{thebibliography}{}
	
\end{thebibliography}
\end{document}
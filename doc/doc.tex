\documentclass{llncs}
%
\usepackage{makeidx}  % allows for indexgeneration
\usepackage[utf8]{inputenc}
\begin{document}
\mainmatter              % start of the contributions
%
\title{Study of Connection Patterns in Opportunistic Networks}
%
\author{Antonio Cardace}
%
\institute{Universit\`a di Bologna, Dipartimento di Scienze dell'Informazione
\email{antonio.cardace2@studio.unibo.it}\\Matricola : 0000}
\maketitle
%
\begin{abstract}
A mobile ad hoc network (MANET) is a continuously self-configuring, infrastructure-less network of mobile devices connected without wires. Each   device/node in a MANET is free to move independently in any direction, and   will therefore change its links to other devices frequently. The main problem in a MANET is to maintain connectivity. A simulation model has been implemented in NetLogo to analyze the behavior of the network according to certain varying properties, the results comprehend some common evaluation metrics calculated on many experiments in which different strategies for handling connections have been used.
\end{abstract}

\section{The System}

\end{document}

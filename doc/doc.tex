\documentclass{llncs}
%
\usepackage{makeidx}  % allows for indexgeneration
\usepackage[utf8]{inputenc}
\usepackage{graphicx}
\usepackage{epstopdf}
\usepackage{float}
\usepackage[export]{adjustbox}
%
\newcommand{\exedout}{%
	\rule{0.8\textwidth}{0.5\textwidth}%
}
\begin{document}
\mainmatter              % start of the contributions
%
\title{Study of Connection Patterns in Opportunistic Networks}
%
\author{Antonio Cardace}
%
\institute{Universit\`a di Bologna, Dipartimento di Scienze dell'Informazione
\email{antonio.cardace2@studio.unibo.it}\\Matricola : 0000738443}
\maketitle
%
\begin{abstract}
A mobile ad hoc network (MANET) is a continuously self-configuring, infrastructure-less network of mobile devices connected without wires. Each   device/node in a MANET is free to move independently in any direction, and   will therefore change its links to other devices frequently. The main problem in a MANET is to maintain connectivity. A simulation model has been implemented in NetLogo to analyze the behavior of the network according to certain varying properties, the results comprehend some common evaluation metrics calculated on many experiments in which different strategies for handling connections have been used.
\end{abstract}

\section{The Model}
\subsection{Base Model}
The model which has been used defines a type of agent : a node, every node moves independently and randomly. Two nodes can communicate when there is a link between them, this link gets created only if the distance between the two nodes is less or equal than $r$ which is the coverage area of a node. Obviously when two connected nodes move beyond their coverage area, the link between them is removed.\\\\
%
In addition to the coverage area of a node there is another constraint to be satisfied in order for a link to be established, both nodes must have a degree which is less than the maximum possible degree.\\\\
%
A node also has a defined speed of movement, according to this parameter we know how many steps the node can move forward in just a unit of time.\\\\
%
The last property of the model is the strategy the nodes adopt for handling connection, which defines the behavior when events such as "establish new connection" or "replace a connection" occur.\\\\
%
The area on which this simulation model has been deployed is a torus of variable size.\\
%
\subsection{Extensions to the Base Model}
The base model which has just been described has been extended with few properties.\\
\begin{itemize}
	\item \textbf{Node speed} : every node can have its own speed parameter. When the global parameter $s$ (node speed) is set a node set its own speed parameter which is chosen randomly and uniformly between 1 and $s$;\\
	%
	\item \textbf{Node maximum degree} : every node can have its own maximum degree parameter. When the global parameter $D_{m}$ (maximum degree) is set a node set its own maximum degree parameter which is chosen randomly and uniformly between 1 and $D_{m}$.
\end{itemize}
%
\subsection{Connection Handling Strategies}
%
The base rule is that every node establish a new connection when possible, therefore the different strategies define how a node behaves when it has reached the maximum number of connections and a new node gets discovered (it means that the new node enters in its coverage area), this implies that some decision has to be made about whether removing an older connection in order to establish the new one and according to what criterion this must be done.\\\\
%
These are the different connection handling strategies present in the NetLogo implementation of the model and which are therefore available for the simulation experiments.\\\\
The following strategies define whether to remove a connection and how a node decides which connection to remove before establishing a new one:\\
\begin{enumerate}
	\item \textbf{Random-kill} : this strategy always removes an existing random connection when a new node gets discovered;
	\\
	%
	\item \textbf{Max-degree-kill} : When a new node gets discovered this strategy always removes the connection with the node having the maximum degree among a node's neighbors;
	\\
	%
	\item \textbf{No-bridge-kill (random)} : this strategy tries to remove a random connection which is not a bridge, this implies that removing this connection the connected component to which the node belongs is not broken in 2 smaller components. When a node only has bridge-connections it removes a random connection;
	\\\\
	%
	\item \textbf{No-bridge-kill (max-degree)} : this strategy tries to remove a random connection which is not a bridge, when a node only has bridge-connections it removes the connection with the node with the maximum degree;
	\\
	%
	\item \textbf{No-bridge-kill (most-distant)} : this strategy tries to remove a random connection which is not a bridge, when a node only has bridge-connections it removes the connection with the node which is the most distant;
	\\
	%
	\item \textbf{No-bridge-kill} : this strategy tries to remove a random connection which is not a bridge, if all connections are bridges the new connection is not established;
	\\
	%
	\item \textbf{Most-distant-kill} : this strategy always removes the connection with the most distant node in order to be able to connect to a new node;
	\\
	%
	\item \textbf{Most-distant-no-bridge-kill} : this strategy tries to remove a non-bridge connection with the most distant node, if every connection is a bridge then the new connection is not established;
	\\
	%
	\item \textbf{Max-degree-no-bridge-kill} : this strategy tries to remove a non-bridge connection with the node with the maximum degree, if every connection is a bridge then the new connection is not established;
	%
	\\\\
	All the "no-bridge" strategies are \underline{theoretical} strategies because in a MANET there's no centralized vision of what is happening, therefore these kind of strategies in reality are not possible, nevertheless I included them as being "optimal" strategies because they were of interest to me.
	%
	\\\\
	All the connection handling strategies which have just been described will be used in the following experiments which have been performed with the NetLogo software.
\end{enumerate}
%
\section{Evaluation Metrics}
For the purpose of evaluating the simulation results of the implementation of the model some metrics have been considered and computed on each performed experiment.\\
%
\subsection{Connectivity}
One of the evaluation metric which has been taken into account is the growth over time of the biggest connected component, this has a strong meaning for the network we obtain because what it basically means is how many node is the giant component made of and therefore how many nodes are at a given time reachable in terms of communication.\\\\
%
As a matter of fact the size of the biggest component present in the network has been defined as being the value of \textbf{connectivity} of the network, which defines the probability that if we randomly pick two nodes of the network there exists a path between them through which they are capable of communicating with each other.\\\\
%
Given this definition of connectivity this is the metric of the network we obtain which we most care and pursue to maximize (the greater this value is the better the connectivity is).
\subsection{Edge Density}
This evaluation metric takes into consideration how many edges are present over the possible number of edges.\\
This metric is computed a bit differently from the standard one because in this model we have a constraint on the number of edges which each node can have (the maximum degree).\\\\
%
The edge density $E_{D}$ is computed as follows:\\\\
%
\[ E_{D} = \frac{2 * E_{c} }{ N_{c} * D_{m} }  \]
\\
where $E_{c}$ is the number of edges present in the network and $N_{c}$ is the number of nodes present in the network.
\subsection{Bridges}
An edge in a graph is a bridge if deleting it increases the number of components of the graph.\\\\
%
In this evaluation metric the number of bridges over the number of edges is taken into account (to normalize the value between 0-1), for the purpose of examining the network we obtain during the experiments we are only interested in calculating the percentage of bridges present in the giant component.\\\\
%
This metric is computed as follows : \\
%
\[ B_{gc} = \frac{Bn_{gc}}{E_{gc}} \]
%
\\
where $Bn_{gc}$ is the number of bridges present in the giant component.\\\\
%
The percentage of bridges is defined as the fault-tolerance of the network, the lower the number of bridges is the higher is the resilience of the network, as a matter of fact the metric $B_{gc}$ represents the probability of breaking the giant component in two distinct components removing a connection picked randomly, so as stated before the lower this probability is the better is the network.
\subsection{Degree Distribution}
The last metric which is the degree distribution doesn't help us in evaluating how good is the network we obtain with a certain configuration, but it is still of interest because we may want to know to what law the degree distribution obeys.\\\\
As a matter of fact it may be curious to find out that the degree distribution when the network is fully connected might obey either a power law, therefore there are some nodes which are hubs, or a normal distribution.
%
\section{Experiments}
%
In this section there are the results of the experiments with different configurations of the parameters of the model we have described so far.\\\\
%
Since the range of the parameters which characterize the model is too wide we will make some assumptions on the value of some them as to study the parameters which are more of interest with the objective to find relations between them and if possible to find the tipping points of these properties to have a fully connected network.\\\\
%
Throughout the experiments the following parameters will be set to the following values : \\
%
\begin{itemize}
	\item $r$ (radius) = 15%
	\item $s$ (node speed) = 5%
\end{itemize}
%
These values are relative to the dimension of the torus on which the simulation model is deployed.
\\\\
All the plots shown in this section have been made with the software R using the library ggplot2 and a smoothing algorithm has been applied to all the datasets used for the graphs (except for the degree distribution plot).
\\\\
During the following sections not all the strategies will be shown, in fact there are only the main ones (No-bridge-kill, Most-distant-kill, Max-degree-kill, Random-kill), this has been done because all the "hybrid" strategies perform almost the same as their "pure" version and don't exhibit therefore interesting behaviors.
\subsection{Connectivity experiments}
In these experiments the values of the number of nodes in the system and their maximum degree has been varied to observe how it affects the connectivity of the network we obtain.
\begin{figure}[h!]
	\begin{minipage}{0.5\textwidth}
		\includegraphics[scale=0.6,right]{plots/p2-Connectivity_Max-degree-kill}
		\caption{Connectivity experiment - 1/2}
		\label{fig:exp1}
	\end{minipage}\hfill
	\begin{minipage}{0.5\textwidth}
		\centering
		\includegraphics[scale=0.6,left]{plots/p1-Connectivity_Max-degree-kill}
		\caption{Connectivity experiment - 3/4}
		\label{fig:exp2}
	\end{minipage}
\end{figure}
\\\\\\\\\\
In Fig.~\ref{fig:exp1} and Fig~\ref{fig:exp2} there are the first four experiments in which we compare the performance of the following strategies:
\begin{itemize}
	\item Random-kill;
	\item Max-degree-kill;
	\item No-bridge-kill (theoretical);
	\item Most-distant-kill;
\end{itemize}
%
We can observe that all the strategies apart from the "No-bridge-kill" one behave similarly.
\\\\
In fact we can deduce from the plots that the amount of nodes is almost irrelevant for the connectivity, as a matter of fact in all the strategies when the maximum degree is set to the five the connectivity doesn't go beyond the 45-50\%, but when we go from $D_{m}=5$ to $D_{m}=15$ we have a sudden change, in fact we can say that between 5 and 15 ($D_{m}$) there's a tipping point value for the connectivity metric.\\\\
%
From the plots we can observe that increasing $D_{m}$ up to 15 or more the connectivity goes up to 80\% and the more we increase this value the higher the connectivity is, until reaching almost 100\% for the whole simulation time.
\\\\
These results are quite predictable because even if the radius is fixed and equals to  15\% when we increase the number of nodes they end up in filling up the whole area on which they are deployed and if the number of possible connection per node (the maximum degree) is sufficient then we can achieve full connectivity.\\
From these results we can see that this last parameter is the most relevant (compared to the nodes number, because the others are fixed) because independently of the handling connection strategy adopted (even the "Random-kill" performs quite well) the networks we obtain are all quite good in terms of connectivity when the maximum degree value is set to 15.
\\\\
The "No-bridge-kill" obviously stands out in terms of connectivity performance because is an optimal behavior, as a matter of fact adopting this strategy the nodes remove a connection in order to replace it with another one only if this connection doesn't break the connected component to which the node belongs in more components, therefore it clearly performs really well, in fact it achieves a value of connectivity higher than the 80\% in most cases (all of them actually, apart from when the nodes are 70) even when $D_{m}=5$.\\\\
Unfortunately we have to point out again that this strategy is only theoretical and not implementable in reality, because in a MANET network every node is independent and there is no central authority having the information required to be capable of computing whether a connection is a bridge or not.
%
%
\subsection{Edge density experiments}
In these experiments the values of the number of nodes in the system and their maximum degree has been varied to observe how it affects the edge density of the network we obtain.
\begin{figure}[h!]
	\begin{minipage}{0.5\textwidth}
		\includegraphics[scale=0.6,right]{plots/p2-Edge-density_Max-degree-kill}
		\caption{Edge density experiment - 1/2}
		\label{fig:edge1}
	\end{minipage}\hfill
	\begin{minipage}{0.5\textwidth}
		\centering
		\includegraphics[scale=0.6,left]{plots/p1-Edge-density_Max-degree-kill}
		\caption{Edge density experiment - 3/4}
		\label{fig:edge2}
	\end{minipage}
\end{figure}
%
\\\\
From the plots in Fig~\ref{fig:edge1} and Fig~\ref{fig:edge2} we can observe that no strategy is really effective in increasing the density of the network, apart from the "No-bridge-kill" one which increases this value (compared to the other strategies) when the nodes in the system are 70 and $D_{m}=5$, this is understandable because of the goodness of the "No-bridge-kill" at keeping bridge connections intact and for the fact that we roughly have the same number of connections over all the experiments, therefore the lower $D_{m}$ is the higher the edge density is, this is a direct consequence of the how the metric is computed.
%
\subsection{Bridges experiments}
In these experiments the values of the number of nodes in the system and their maximum degree has been varied to observe how it affects the value of $B_{gc}$ of the network we obtain.
\begin{figure}[h!]
	\begin{minipage}{0.5\textwidth}
		\includegraphics[scale=0.6,right]{plots/p2-Bridges_Max-degree-kill}
		\caption{Bridge percentage - 1/2}
		\label{fig:br1}
	\end{minipage}\hfill
	\begin{minipage}{0.5\textwidth}
		\centering
		\includegraphics[scale=0.6,left]{plots/p1-Bridges_Max-degree-kill}
		\caption{Bridge percentage experiment - 3/4}
		\label{fig:br2}
	\end{minipage}
\end{figure}
%
\\\\
The plots in Fig.~\ref{fig:br1} and Fig.~\ref{fig:br2} show the percentage of bridges in the different networks obtained during the experiments, to recap the value of this metric has been defined as the fault-tolerance of network since $B_{gc}$ is the probability to break the giant component in more components if we remove a random connection from it, therefore the lower it is the more resilient the network is.
\\\\
Obviously the best handling connection strategy at minimizing this value is the "No-bridge-kill", but since it is so obvious is not interesting.\\
What is actually interesting is the behavior of the network independently of the strategy adopted, as a matter of fact we can observe that if we increase $D_{m}$ up to 15 then all the strategies keep the value of $B_{gc}$ below the 12-13\%, therefore we might deduce that there is a tipping point between the 5\% and 15\%, and in fact the tipping, as we can see from Fig.~\ref{fig:tip1} is actually 6\%.
\begin{figure}[h!]
	\includegraphics[scale=0.5]{plots/bridge_tipping}
	\caption{tipping point value - Bridges}
	\label{fig:tip1}
\end{figure}
\section{Conclusions} 
% ---- Bibliography ----
%
\begin{thebibliography}{}
	
\end{thebibliography}
\end{document}
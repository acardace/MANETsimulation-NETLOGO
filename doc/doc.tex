\documentclass{llncs}
%
\usepackage{makeidx}  % allows for indexgeneration
\usepackage[utf8]{inputenc}
\begin{document}
\mainmatter              % start of the contributions
%
\title{Study of Connection Patterns in Opportunistic Networks}
%
\author{Antonio Cardace}
%
\institute{Universit\`a di Bologna, Dipartimento di Scienze dell'Informazione
\email{antonio.cardace2@studio.unibo.it}\\Matricola : 0000738443}
\maketitle
%
\begin{abstract}
A mobile ad hoc network (MANET) is a continuously self-configuring, infrastructure-less network of mobile devices connected without wires. Each   device/node in a MANET is free to move independently in any direction, and   will therefore change its links to other devices frequently. The main problem in a MANET is to maintain connectivity. A simulation model has been implemented in NetLogo to analyze the behavior of the network according to certain varying properties, the results comprehend some common evaluation metrics calculated on many experiments in which different strategies for handling connections have been used.
\end{abstract}

\section{The Model}
\subsection{Base Model}
The model which has been used defines a type of agent : a node, every node moves independently and randomly. Two nodes can communicate when there is a link between them, this link gets created only if the distance between the two nodes is less or equal than $r$ which is the coverage area of a node. Obviously when two connected nodes move beyond their coverage area, the link between them is removed.\\\\
%
In addition to the coverage area of a node there is another constraint to be satisfied in order for a link to be established, both nodes must have a degree which is less than the maximum possible degree.\\\\
%
A node also has a defined speed of movement, according to this parameter we know how many steps the node can move forward in just a unit of time.\\\\
%
The last property of the model is the strategy the nodes adopt for handling connection, which defines the behavior when events such as "establish new connection" or "replace a connection" occur.\\\\
%
The area on which this simulation model has been deployed is a torus of variable size.\\
%
\subsection{Extensions to the Base Model}
The base model which has just been described has been extended with few properties.\\
\begin{itemize}
	\item \textbf{Node speed} : every node can have its own speed parameter. When the global parameter $s$ (node speed) is set a node set its own speed parameter which is chosen randomly and uniformly between 1 and $s$;\\
	%
	\item \textbf{Node maximum degree} : every node can have its own maximum degree parameter. When the global parameter $D_{m}$ (maximum degree) is set a node set its own maximum degree parameter which is chosen randomly and uniformly between 1 and $D_{m}$.
\end{itemize}
%
\subsection{Connection Handling Strategies}
%
The base rule is that every node establish a new connection when possible, therefore the different strategies define how a node behaves when it has reached the maximum number of connections and a new node gets discovered (it means that the new node enters in its coverage area), this implies that some decision has to be made about whether removing an older connection in order to establish the new one and according to what criterion this must be done.\\\\
%
These are the different connection handling strategies present in the NetLogo implementation of the model and which are therefore available for the simulation experiments.\\\\
The following strategies define whether to remove a connection and how a node decides which connection to remove before establishing a new one:\\
\begin{enumerate}
	\item \textbf{Random-kill} : this strategy always removes an existing random connection when a new node gets discovered;
	\\
	%
	\item \textbf{Max-degree-kill} : When a new node gets discovered this strategy always removes the connection with the node having the maximum degree among a node's neighbors;
	\\
	%
	\item \textbf{No-bridge-kill (random)} : this strategy tries to remove a random connection which is not a bridge, this implies that removing this connection the connected component to which the node belongs is not broken in 2 smaller components. When a node only has bridge-connections it removes a random connection;
	\\\\
	%
	\item \textbf{No-bridge-kill (max-degree)} : this strategy tries to remove a random connection which is not a bridge, when a node only has bridge-connections it removes the connection with the node with the maximum degree;
	\\
	%
	\item \textbf{No-bridge-kill (most-distant)} : this strategy tries to remove a random connection which is not a bridge, when a node only has bridge-connections it removes the connection with the node which is the most distant;
	\\
	%
	\item \textbf{No-bridge-kill} : this strategy tries to remove a random connection which is not a bridge, if all connections are bridges the new connection is not established;
	\\
	%
	\item \textbf{Most-distant-kill} : this strategy always removes the connection with the most distant node in order to be able to connect to a new node;
	\\
	%
	\item \textbf{Most-distant-no-bridge-kill} : this strategy tries to remove a non-bridge connection with the most distant node, if every connection is a bridge then the new connection is not established;
	\\
	%
	\item \textbf{Max-degree-no-bridge-kill} : this strategy tries to remove a non-bridge connection with the node with the maximum degree, if every connection is a bridge then the new connection is not established;
	%
	All the connection handling strategies which have just been described will be used in the following experiments which have been performed with the NetLogo software.
	
\end{enumerate}
%
\section{Evaluation Metrics}
For the purpose of evaluating the simulation results of the implementation of the model some metrics have been considered and computed on each performed experiment.\\
%
\subsection{Connectivity}
One of the evaluation metric which has been taken into account is the growth over time of the biggest connected component, this has a strong meaning for the network we obtain because what it basically means is how many node is the giant component made of and therefore how many nodes are at a given time reachable in terms of communication.\\\\
%
As a matter of fact the size of the biggest component present in the network has been defined as being the value of \textbf{connectivity} of the network, which defines the probability that if we randomly pick two nodes of the network there exists a path between them through which they are capable of communicating with each other.\\\\
%
Given this definition of connectivity this is the metric of the network we obtain which we most care and pursue to maximize (the greater this value is the better the connectivity is).
\subsection{Edge Density}
This evaluation metric takes into consideration how many edges are present over the possible number of edges.\\
This metric is computed a bit differently from the standard one because in this model we have a constraint on the number of edges which each node can have (the maximum degree).\\\\
%
The edge density $E_{D}$ is computed as follows:\\\\
%
\[ E_{D} = \frac{2 * E_{gc} }{ N_{gc} * D_{m} }  \]
\\
where $E_{gc}$ is the number of edges present in the biggest/giant component and $N_{gc}$ is the number of nodes present in the biggest/giant component.
\subsection{Bridges}
An edge in a graph is a bridge if deleting it increases the number of components of the graph.\\\\
%
In this evaluation metric the number of bridges over the number of edges is taken into account (to normalize the value between 0-1), for the purpose of examining the network we obtain during the experiments we are only interested in calculating the percentage of bridges present in the giant component.\\\\
%
This metric is computed as follows : \\
%
\[ B_{gc} = \frac{2 * Bn_{gc}}{N_{gc} * D_{m} } \]
%
\\
where $Bn_{gc}$ is the number of bridges present in the giant component.
\subsection{Degree Distribution}
%
\section{Experiments}
%
\section{Conclusions}
% ---- Bibliography ----
%
\begin{thebibliography}{}
	
\end{thebibliography}
\end{document}